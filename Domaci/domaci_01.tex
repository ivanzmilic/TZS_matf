\documentclass[12pt]{article}
\usepackage{amsmath}
\usepackage{amssymb}
\usepackage{amsfonts}
\usepackage[colorlinks=true]{hyperref}
\usepackage{graphicx}

\title{Prvi Doma\'{c}i zadatak}
\author{TZS}
\date{\today}

\begin{document}
\maketitle

U izradi doma\'{c}eg zadatka se mo\v{z}ete konsultovati medjusobno i sa mnom. Svaki doma\'{c}i koji predajete, medjutim, mora biti samostalno napisan. 

\textbf{Rok za predaju ovog doma\'{c}eg zadatka je ponedeljak 22.12.2025. Prvi zadatak nosi 8 poena a ostala tri po 4 poena.}

\section*{Zadatak 1}

Razmatrajmo polubeskona\v{c}nu, plan-paralelnu atmosferu u kojoj funkcija izvora na nekoj, referentnoj talasnoj du\v{z}ini zavisi od opti\v{c}ke dubine kao:
\begin{equation}
S = a + b\tau
\end{equation}
Ovo je poznato kao Milne-Eddingtonova (ili Milne-Barbier-Unsold approksimacija) i na osnovu nje mozemo da dodjemo do raznih korisnih relacija koje nam omogu\'{c}avaju da bolje razumemo zvezdane atmosfere. Medjutim, u zvezdanim atmosferama bi imalo vi\v{s}e smisla koristiti $\ln \tau$ kao skalu dubine. Pretpostavimo, dakle, da na\v{s}a funkcija izvora zavisi od referentne opti\v{c}ke dubine kao:
\begin{equation}
S = a + b\ln\tau
\end{equation}


\begin{itemize}
    \item Re\v{s}iti jedna{v}cinu prenosa zra\v{c}enja na referentnoj talasnoj du\v{z}ini, tj. izraziti izlazni intenzitet preko konstanti $a, b$. Ovaj intenzitet \'{c}emo zvati $I^+$. Napomena: Integral koji se dobija nije mogu\'{c}e re\v{s}iti analiti\v{c}ki, tako da morate iskoristiti npr. Mathematicu, Wolfram Alpha ili sli\v{c}no.
    
    \item Ova pretpostavka ima jedan konceptualan problem a to je da na malim opti\v{c}kim dubinama, $\ln \tau$ ide u $-\infty$ pa, bez obzira koliko je koeficijent $b$ mali, funkcija izvora bi postala negativna. To mo\v{z}emo da popravimo tako \v{s}to \'{c}emo pretpostaviti da je funkcija izvora paraboli\v{c}na funkcija od $\ln \tau$:
    \begin{equation}
        S = a + b\ln \tau + c \ln^2 \tau
    \end{equation}
    Re\v{s}iti jedna\v{c}inu prenosa za ovakav oblik funkcije izvora. 
    
    \item Pretpostavimo (va\v{z}i za relativno velike talasne du\v{z}ine) da je funkcija izvora propoprcionalna Temperaturi. Jednostavnosti radi uzmimo da je konstanta proporcionalnosti jednaka jedan. Na\'{c}i $a, b, c$ tako da je $T(\ln\tau=0) = 6000$ (fotosfera), $T(\ln\tau=-7) = 4500$ (tzv. temperaturski minimum), $T(\ln\tau=-14) = 8000$ (hromosfera). Izra\v{c}unaj numeri\v{c}ku vrednost $I^+$. Da li va\v{z}i da je izlazni intenzitet pribli\v{z}no jednak funkciji izvora na $\tau=1$ (tj. $\log \tau = 0$)?

    \item Kakav bi bio izlazni intenzitet na talasnoj du\v{z}ini na kojoj je koeficijent neprozra\v{c}nosti $r_\lambda$ puta ve\'{c}i od referentnog? Skicirajte / isplotujte zavisnost $I^+_\lambda$ od $r_\lambda$ ($r_\lambda >  1$).
    
\end{itemize}

\section*{Zadatak 2}

Pokazati da u izotermalnoj atmosferi, gde vlada sila gravitacije upravljena na dole, pritisak gasa zavisi od visine kao: 
\begin{equation}
p(h) = p_0 e^{-h/H}
\end{equation}
gde je $p_0$ pritisak na povr\v{s}ini. Pokazati da $H$ zavisi od temperature i srednje molekulske mase gasa. Izra\v{c}unati $H$ za Zemlju i za Sun\v{c}evu atmosferu (pretpostaviti da je temperature Sun\v{c}eve atmosfere jednaka efektivnoj temperaturi Sunca, 5800~K).

\section*{Zadatak 3}

Pretpostavimo da teleskopom pre\v{c}nika 1 metar, posmatramo Sunce. Na\v{s} instrument je takav jedan pikselel na slici formiranoj u teleskopu odgovara kvadratu na Sun\v{c}evoj povr\v{s}ini ivice 50~km. Takodje, posmatramo na talasnoj du\v{z}ini 393~nm, u intervalu spektralne \v{s}irine 0.01~nm. Koliko fotona primi jedan piksel na\v{s}eg detektora u sekundi pri ovakvom posmatranju?

\section*{Zadatak 4}

Protuberance (eng: \emph{prominences}) i filamenti su po na\v{s}em trenutnom shvatanju jedni te isti objekti (videti sliku): relativno hladne koncentracije gasa koje pod uticajem magnetnog polja ``vise'' u Sun\v{c}evoj koroni. Filamente vidimo na disku: nevidljivi su u kontinuumu, ali se vide kao tamne ``trake'' na talasnim du\v{z}inama u centru jakih spektralnih linija (npr. H$\alpha$). Protuberance, sa druge strane, se vide iznad Sun\v{c}evog ruba kao svetle formacije u centru jakih spektralnih linija. Ukoliko su posmatra\v{c}ki uslovi izuzetni, mogu se videti i u kontinuumu. Koriste\'{c}i formalizam prenosa zra\v{c}enja, objasniti razliku izmedju protuberanci i filamenata. Pomo\'{c}: Obratite pa\v{z}nju na interakciju izmedju ulaznog intenziteta, opti\v{c}ke dubine (debljine) objekta i funkcije izvora u objektu.

\begin{figure}
\includegraphics[width=\textwidth]{promfil.jpg}
\caption{Levo: primer protuberance. Desno: ista ta protuberanca, koja se vidi kao filament.}
\end{figure}




\end{document}
