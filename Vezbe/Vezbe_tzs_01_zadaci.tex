\documentclass[12pt]{article}
\usepackage{amsmath}
\usepackage{amssymb}
\usepackage{amsfonts}
\usepackage[colorlinks=true]{hyperref}


\title{TZS Ve\v{z}be: 3. Nedelja, 26/10/2022}
\author{Jasmina Horvat, Ivan Mili\'{c}}
\date{\today}

\begin{document}
\maketitle

\section{Zadatak 1}
Pokazati da za ``gas fotona'', ako ih tretiramo kao \v{c}estice koje udaraju o zidove neke zatvorene zapremine, va\v{z}i da: 
\begin{equation}
p_{\rm fotona} = \frac{4\pi}{c} K
\end{equation}
gde je K-integral ($K$) definisan:
\begin{equation}
K = \frac{1}{4\pi}\oint \int_0^{\infty}I_\lambda(\theta,\phi)\cos^2 \theta \sin \theta d \theta d \phi
\end{equation}

\section{Zadatak 2}
Razmatrajmo zvezdanu atmosferu za koju smo pretpostavili da va\v{z}i da je $S=a+b\tau$ (ovde razmatramo samo jednu talasnu du\v{z}inu pa mo\v{z}emo zanemariti zavisnost od talasne du\v{z}ine).
\begin{itemize}
    \item Pokazati da za izlazni intenzitet va\v{z}i da je $I = a +b\mu$ (Uradjeno na \v{c}asu).

    \item Pokazati da isto va\v{z}i i za ulazni i za izlazni intenzitet kada $\tau >> 1$.

    \item Izra\v{c}unati $J, F$ i $K$ u zavisnosti od $a$ i $b$ na povr\v{s}ini i na $\tau >> 1$. Uo\v{c}iti odnos izmedju $K$ i $J$. 
\end{itemize}



\section{Zadatak 3}

Zemljina atmosfera apsorbuje oko 25\% zra\v{c}enja izvora koji je u zenitu.
\begin{itemize}
    \item Koja je ukupna opti\v{c}ka dubina (debljina) Zemljine atmosfere?
    \item Pod pretpostavkom da je temperatura atmosfere konstantna, i da iznosi oko 300\,K, odredite skalu visine za zemljinu atmosferu.
    \item Procenite koncentraciju \v{c}estica vazduha na povr\v{s}ini. 
    \item Iz rezultata prethodnih delova, izvedite prose\v{c}an efikasni presek za apsorpciju na \v{c}esticama vazduha. (Iako je vazduh me\v{s}avina gasova, pretpostavi\'{c}emo da se svi pona\v{s}aju isto.)
    \item Kako se menja opti\v{c}ka dubina koju ``vidi'' svetlost, ako izvor nije u zenitu? Dati opisan odgovor.
\end{itemize}

\end{document}

