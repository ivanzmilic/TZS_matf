\documentclass[12pt]{article}
\usepackage{amsmath}
\usepackage{amssymb}
\usepackage{amsfonts}
\usepackage[colorlinks=true]{hyperref}


\title{TZS Ve\v{z}be: 6. Nedelja, 13/11/2022}
\author{Jasmina Horvat, Ivan Mili\'{c}}
\date{\today}

\begin{document}
\maketitle

\section{Zadatak}
Izvesti Švarcišld-Šusterovo i Edingtonovo rešenje Milneovog problema. 

\section{Zadatak}
Za Milne-Eddingtonovo rešenje, naći temperaturu Sunca na $\tau=1$ i $\tau=0$. 

\section{Zadatak}
Na\'{c}i intenzitet u svim pravcima i na svim optičkim dubinama u atmosferi, ako je u pitanju Milne-Eddingtonova siva atmosfera ($S=a+b\tau$)

\section{Zadatak}
Pokazati da $\Lambda$ operator koji slika funkciju izvora u srednji intenzitet ima slede\'{c}i oblik:
\begin{equation}
J(\tau) = \Lambda [S] = \frac{1}{2} \int_0^{\infty} S(t) E_1(|t-\tau|) dt 
\end{equation} 

\section{Zadatak}
Pod pretpostavkom da je struktura atmosfere u skladu sa Eddingtonovim re\v{s}enjem Milneovog problema, proceniti ja\v{c}inu Balmerovog skoka, ako je efikasna temperatura zvezde 8000 K a $\chi^-_{Balmer} = 10 \chi^+_{Balmer}$, a $\overline{\chi} = \chi^+$. Dovoljno je opisno uraditi zadatak (tj. bez zamene brojki).

\section{Zadatak}
Ako se gas sastoji samo od protona, elektrona i neutralnog vodonika, i ako su date \emph{elektronska} koncentracija gasa (ili elektronski pritisak) i temperatura, izra\v{c}unati koncentraciju protona i neutralnog vodonika, kao i ukupan pritisak gasa. Prodiskutovati razlike u odnosu na slu\v{c}aj kada je bio dat pritisak gasa. 

\section{Zadatak}
Postaviti prethodni zadatak ako se gas sastoji od protona, elektrona, neutralnog vodonika, neutralnog helijuma i jednom i dvaput jonizovanog helijuma. 

\section{Zadatak}
Za plazmu temperature 4000\,K i fotosferskog pritiska, uporediti koncentraciju vodonika na nivou $n=2$ i koncentraciju gvo\v{z}dja (zastupljenost gvo\v{z}dja je oko $3\times 10^{-5}$ u odnosu na vodonik).





\end{document}

