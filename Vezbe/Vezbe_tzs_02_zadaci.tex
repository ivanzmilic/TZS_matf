\documentclass[12pt]{article}
\usepackage{amsmath}
\usepackage{amssymb}
\usepackage{amsfonts}
\usepackage[colorlinks=true]{hyperref}


\title{TZS Ve\v{z}be: 2. Nedelja, 18/10/2022}
\author{Jasmina Horvat, Ivan Mili\'{c}}
\date{\today}

\begin{document}
\maketitle

\section{Zadatak}
Pokazati da za ``gas fotona'', ako ih tretiramo kao \v{c}estice koje udaraju o zidove neke zatvorene zapremine, va\v{z}i da: 
\begin{equation}
p_{\rm fotona} = \frac{4\pi}{c} K
\end{equation}
gde je K-integral ($K$) definisan:
\begin{equation}
K = \frac{1}{4\pi}\oint \int_0^{\infty}I_\lambda(\theta,\phi)\cos^2 \theta d\lambda \sin \theta d \theta d \phi
\end{equation}
Napomena: $p_{\rm fotona}$ se \v{c}esto zove ``pritisak zra\v{c}enja''. 

\section{Zadatak}
Jedna\v{c}ina prenosa zra\v{c}enja (JPZ) ima slede\'{c}i oblik:
\begin{equation}
\frac{dI_\lambda}{ds} = -\chi_\lambda I_\lambda + j_\lambda
\end{equation}
gde su $\chi_\lambda$ i $j_\lambda$ koeficijenti apsorpcije i emisije (koji zavise od talasne du\v{z}ine). Re\v{s}iti JPZ za slu\v{c}ajeve kada imamo samo apsorpciju i samo emisiju i one ne zavise od $s$. Obratiti pa\v{z}nju na grani\v{c}ne uslove! 

\section{Zadatak}
Opti\v{c}ka dubina nekog objekta se defini\v{s}e kao: 
\begin{equation}
\tau_\lambda = \int \chi_\lambda (s) ds
\end{equation}

Opti\v{c}ka dubina Zemljine atmosfere u pravcu $z$ je oko 0.25. 
\begin{itemize}
    \item Pod pretpostavkom da je temperatura atmosfere konstantna, i da iznosi oko 300\,K, odredite skalu visine za zemljinu atmosferu.
    \item Ponoviti ra\v{c}un za Sun\v{c}evu atmosferu ($T \approx 6000$ K). Pretpostaviti da se atmosfera Sunca sastoji od neutralnog vodonika. 
    \item Procenite koncentraciju \v{c}estica vazduha na povr\v{s}ini. 
    \item Iz rezultata prethodnih delova, izvedite prose\v{c}an efikasni presek za apsorpciju na \v{c}esticama vazduha. (Iako je vazduh me\v{s}avina gasova, pretpostavi\'{c}emo da se svi pona\v{s}aju isto.)
    \item Kako se menja opti\v{c}ka dubina koju ``vidi'' svetlost, ako izvor nije u zenitu? Dati opisan odgovor.
\end{itemize}

\section{Zadatak}

Izvesti izraz za bolometrijsku luminoznost zvezde ako znamo kako izgleda $I_\lambda(x,y,z,\theta,\phi)$.

\end{document}

