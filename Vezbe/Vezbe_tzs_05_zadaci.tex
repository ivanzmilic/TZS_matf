\documentclass[12pt]{article}
\usepackage{amsmath}
\usepackage{amssymb}
\usepackage{amsfonts}
\usepackage[colorlinks=true]{hyperref}


\title{TZS Ve\v{z}be: 7. Nedelja, 08/12/2022}
\author{Jasmina Horvat, Ivan Mili\'{c}}
\date{\today}

\begin{document}
\maketitle

\section{Zadatak 1}
Uraditi prvi zadatak sa kolokvijuma:

Recimo da smo izmerili da potamnjenje Sun\v{c}evog diska ka rubu ima slede\'{c}u zavisnost: 
$$I(\mu) = I_0(2/3 + 1/3 \mu).$$ 

Efektivna temperatura (temperatura definisana preko fluksa) Sunca je 5777\,K. Proceniti temperaturu na $\tau=1$ i $\tau=0$. Pretpostavimo da atmosferu Sunca mo\v{z}emo aproksimirati sivom atmosferom.

\section{Zadatak 2}
Pretpostavimo da analiziramo jednu jaku spektralnu liniju. U centru spektralne linije neprozra\v{c}nost atmosfere je $\chi(\lambda_0)$ a u kontinuumu oko linije $\chi(\lambda_c)$. 

Pre svega, ubedite se da je mogu\'{c}e da:
\begin{align}
\lambda_0 &\approx \lambda_c \nonumber \\
\chi(\lambda_0)  &>> \chi(\lambda_c).
\end{align}

Za linije ne mo\v{z}emo da koristimo ukupnu (bolometrijsku) funkciju izvora, $S$, ve\'{c} moramo da koristimo $S_\lambda \approx S(\lambda_0)$. Pokazati da se S br\v{z}e menja sa temperaturom za male talasne du\v{z}ine nego za velike. 

Pokazati da je $S\propto T$ za velike talasne du\v{z}ine. Uporediti funkcije izvora na temperaturama 4000 K i 7000 K, za talasne du\v{z}ine 200\,nm i 1000\,nm, pod pretpostavkom da je funkcija izvora Plankova funkcija.

\section{Zadatak 3}
Za gas tipi\v{c}ne fotosferske temperature (oko 6000 K), izra\v{c}unati srednju brzinu kojom se kre\'{c}u atomi vodonika i gvo\v{z}dja. Uporediti to sa brzinom elektrona i sa brzinom zvuka u fotosferi.

\section{Zadatak 4}
Pod pretpostavkom da je atmosfera Sunca pribli\v{z}no izotermna, pokazati da koncentracija \v{c}estica (tj. pritisak) opada eksponencijalno sa visinom i odrediti skalu visina. 

\section{Zadatak 5}
Koliko iznad fotosfere (tj. iznad $\tau_c=0$) opti\v{c}ka dubina u kontinuumu opadne na $\tau_c = 10^{-3}$. Pretpostaviti da neprozra\v{c}nost zavisi samo od pritiska gasa. 

\section{Zadatak 6}
Ukoliko bi apsorpcioni profil spektralne linije bio Gausijan sa standardnom devijacijom koja je proporcionalna srednjoj brzini \v{c}estica i centralnoj talasnoj du\v{z}ini preko:
\begin{equation}
\sqrt{2} \sigma = \Delta \lambda_D = \frac{\sqrt{2kT/m}}{c} \lambda_0
\end{equation}
Izra\v{c}unati koliko daleko (u jedinicama talasnih du\v{z}ina) od centra linije neprozra\v{c}nost opadne za faktor od 1000.

\end{document}

