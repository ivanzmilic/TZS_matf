\documentclass[12pt]{article}
\usepackage{amsmath}
\usepackage{amssymb}
\usepackage{amsfonts}
\usepackage[colorlinks=true]{hyperref}


\title{TZS Ve\v{z}be: 7. Nedelja, 19/11/2022 \\
Razni zadaci i priprema za kolokvijum}
\author{Jasmina Horvat, Ivan Mili\'{c}}
\date{\today}

\begin{document}
\maketitle

\section*{Zadatak 1}
Recimo da smo izmerili da potamnjenje Sun\v{c}evog diska ka rubu ima slede\'{c}u zavisnost: 
$$I(\mu) = I_0(2/3 + 1/3 \mu).$$ 

Efektivna temperatura (temperatura definisana preko fluksa) Sunca je 5777\,K. Proceniti temperaturu na $\tau=1$ i $\tau=0$. Pretpostavimo da atmosferu Sunca mo\v{z}emo aproksimirati sivom atmosferom.


\section*{Zadatak 2}
Pod pretpostavkom da je atmosfera Sunca pribli\v{z}no izotermna, pokazati da koncentracija \v{c}estica (tj. pritisak) opada eksponencijalno sa visinom i odrediti skalu visina. Koliko iznad fotosfere (tj. iznad $\tau=0$) opti\v{c}ka dubina u kontinuumu opadne na $\tau = 10^{-3}$? Pretpostaviti da neprozra\v{c}nost zavisi samo od pritiska gasa. 

\section*{Zadatak 3}
Temperatura fotosfere je 5777\,K, a korone, recimo, 2 miliona K. U vidljivom domenu ne vidimo koronu, ali je u UV vidimo. Kolika mora da bude opti\v{c}ka dubina korone da bismo videli koronu iznad Sun\v{c}evog diska u UV (npr, na $\lambda = 21.1$\,nm). Koliko najvi\v{s}e mo\v{z}e da bude da opti\v{c}ka dubina u vidljivom delu spektra, a da je i dalje ne vidimo?

\section* {Zadatak 4}
Izvesti difuznu aproksimaciju: $I(\tau, \mu) = S(\tau) + \mu \frac{dS}{d\tau}$, i napisati fluks preko izvoda funkcije izvora sa opti\v{c}kom dubinom. Da li je mogu\'{c}e da se temperatura pove\'{c}a, a fluks smanji?

\section* {Zadatak 5}
Re\v{s}i Milneov problem koriste\'{c}i aproksimaciju dva zraka (two stream approximation), gde razmatramo samo jedan ulazni i izlazni pravac za zra\v{c}enje, ali pretpostavljamo da njihovi pravci nisu $\mu = 1$ i $\mu=-1$, ve\'{c} $\pm 1/\sqrt{3}$. 


\end{document}

