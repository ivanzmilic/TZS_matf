\documentclass[12pt]{article}
\usepackage{amsmath}
\usepackage{amssymb}
\usepackage{amsfonts}
\usepackage[colorlinks=true]{hyperref}
\usepackage{graphicx}

\title{Prvi Doma\'{c}i zadatak}
\author{TZS}
\date{\today}

\begin{document}
\maketitle

U izradi doma\'{c}eg zadatka se mo\v{z}ete konsultovati medjusobno i sa mnom. Svaki doma\'{c}i koji predajete, medjutim, mora biti samostalno napisan. 

\textbf{Rok za predaju ovog doma\'{c}eg zadatka je petak 10.01.2025. Doma\'{c}i nosi 20 poena.}

\section*{Zadatak 1}

Razmatrajmo Milne-Eddingtonovu sivu atmosferu, dakle: 
\begin{equation}
S = a + b\tau.
\end{equation}
\begin{enumerate}
    \item Pokazati da za $\tau<<1$ funkcija $I(\mu)$ ima diskontinuitet u $\mu=0$. 

    \item Pokazati da $J = 3K$ ne va\v{z}i (za bilo koje $a,b$) na povr\v{s}ini atmosfere. 

    \item Pokazati da $J=3K$ va\v{z}i na velikim dubinama ($\tau >> 1$).

    \item Izraziti $F$ (astrofizi\v{c}ki fluks) na povr\v{s}ini preko $a$ i $b$. 

    \item Izraziti $F$ na velikim dubinama preko $a$ i $b$. 

    \item Izjedna\v{c}ite prethodna dva rezultata da dobijete: 

    \begin{equation}
    S = \frac{3}{4}F (\frac{2}{3} + \tau)
    \end{equation}

    \item Na osnovu svega navedenog objasnite za\v{s}to M-E atmosfera, striktno re\v{c}eno, ne mo\v{z}e da zadovolji Milneov problem. 

    \item Re\v{s}ite Milneov problem metodom diskretnih ordinata, za jedan ulazni i jedan izlazni pravac i odredite vrednost funkcije izvora na povr\v{s}ini atmosfere. Za diskretne ordinate \'{c}emo uzeti Gausovu kvadraturu, gde je $\mu = \pm \frac{1}{\sqrt(3)}$.
\end{enumerate}
\end{document}
