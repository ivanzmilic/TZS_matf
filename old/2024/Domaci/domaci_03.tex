\documentclass[12pt]{article}
\usepackage{amsmath}
\usepackage{amssymb}
\usepackage{amsfonts}
\usepackage[colorlinks=true]{hyperref}
\usepackage{graphicx}
%\usepackage[T1,T2A]{fontenc}
%\usepackage[utf8]{inputenc}
\usepackage[english,serbian]{babel}
%\selectlanguage{serbian}

\usepackage{parskip}
\setlength{\parskip}{5pt}

\title{Prvi Doma\'{c}i zadatak}
\author{TZS}
\date{\today}

\begin{document}
\maketitle

U izradi doma\'{c}eg zadatka se mo\v{z}ete konsultovati medjusobno i sa mnom. Svaki doma\'{c}i koji predajete, medjutim, mora biti samostalno napisan. 

\textbf{Rok za predaju ovog doma\'{c}eg zadatka je petak 31.12.2025. 23:59.}

\section{Setup i atmosferski model}

Cilj ovog zadatka je da sintetizujete profil spektralne linije u lokalnoj termodinamičkoj ravnoteži (LTE) iz datog atmosferskog modela (u ovom slučaju poznatog FALC modela iz rada Averett et al.\ 1993). Izračunaćete populacije atomskih nivoa koristeći Bolcmanovu i Sahinu jednačinu, zatim neprozračnost i funkciju izvora, i konačno rešiti JPZ. U procesu ćete takođe implementirati realističnu neprozračnost u kontinuumu.

Koristite dati jednodimenzioni atmosferski model dat u fajlu \texttt{falc71.dat}. Učitajte ga pomoću \texttt{numpy.loadtxt} ili slične funkcije. Model ima više kolona, od kojih nam trebaju:

\begin{itemize}
    \item[0:] Log optičke dubine u kontinumu na 500\,nm: $\log \tau_{500}$
    \item[1:] Visina $z$ u cm
    \item[2:] Temperatura $T(z)$ u K
    \item[3:] Pritisak gasa $P_{\mathrm{gas}}(z)$ u dyn/cm$^2$
    \item[4:] Elektronski pritisak $n_e(z)$ u dyn/cm$^2$
\end{itemize}

\textbf{Obratite pažnju na jedinice!}

\section{Zadatak}

\subsection{Korak 1: Jonizacija vodonika}

Kao uvod, iskoristite Sahinu jednačinu \textbf{samo za vodonik} (kao na času) da izračunate jonizaciono stanje vodonika kao funkciju visine. Iz toga ćete dobiti i elektronsku koncentraciju. Uporedite dobijene rezultate sa vrednostima iz modela (plotujte u log skali!).

Sahina jednačina glasi (uvek proverite ovo):

\begin{equation}
    \frac{n_{j+1} n_e}{n_j}
    = \frac{2 Z_{j+1}}{Z_j}
      \left( \frac{2\pi m_e kT}{h^2} \right)^{3/2}
      \exp\left( -\frac{E_j}{kT} \right).
\end{equation}

Ovde je $Z$ particiona funkcija. U dubokim slojevima dobićete dobro slaganje sa modelom, ali u višim slojevima naši rezultati pokazuju pad elektrona ispod vrednosti datih u FALC modelu.

\textbf{Zašto se ovo dešava?}

\emph{Napomena: još uvek ne objašnjavamo zašto su naši rezultati iznad modela u najvišim slojevima.}

\subsection{Korak 2: Dodavanje magnezijuma kao dodatnog donora elektrona}

Da bismo popravili rezultate u srednjoj atmosferi, uvodimo dodatni element: magnezijum (pomnožićem njegovu zastupljenost sa 2 da obračunamo i doprinos gvožđa, umesto da rešavamo problem za još jedan element).

Ponovite Saha proračun za sistem H i Mg.  Ovo sada zahteva rešavanje \textbf{nelinearnog sistema jednačina} pomoću, na primer, \texttt{scipy.optimize.fsolve}.

Koristite sledeće parametre:

\begin{itemize}
    \item Energija jonizacije Mg\,I $\rightarrow$ Mg\,II: 7.646 eV
    \item Energija jonizacije Mg\,II $\rightarrow$ Mg\,III: 15.035 eV
    \item Zastupljenost Mg: $3.8\times 10^{-5}$  
          \textbf{Pomnožiti sa 2 da se uračuna i Fe}
    \item Particione funkcije: $Z_{\mathrm{Mg\,I}}=2$, $Z_{\mathrm{Mg\,II}}=1$, $Z_{\mathrm{Mg\,III}}=1$
\end{itemize}

Ponovo izračunajte elektronsku gustinu. Trebalo bi da dobijete znatno bolje slaganje sa modelom. Možete dodatno da varirate zastupljenost Mg ili particionu funkciju za Mg+ da dobijete još bolje rešenje.

\textbf{Napomena:} Često je dovoljno uzeti samo Mg\,I–Mg\,II i zanemariti Mg\,III.

\subsection{Korak 3: Neprozračnost i intenzitet u kontinuumu}

Pre sinteze profila linije potrebno je odrediti \textbf{neprozračnost u kontinuumu} i onda rešiti JPZ za kontinuum.

Dominantni izvori neprozračnosti u vidljivom domenu su negativni jon vodonika $\mathrm{H}^-$ i neutralni vodonik.  Neprozračnost u kontinuumu definišemo definišemo kao:

\begin{equation}
    \chi_\lambda^{\mathrm{cont}}
    = \chi_\lambda^{\mathrm{H^-}} + \chi_\lambda^{\mathrm{H}}.
\end{equation}


Da biste obračunali ova dva doprinosa, potrebno je da iskoristite jednačine koje se pojavljuju knjizi Gray-a i opisuju neprozračnost negativnog jona vodonika i neutralnog vodonika. Ako nemate pristup knjizi, javite se meni.

\subsubsection*{Optička dubina u kontinuumu}

Kao i uvek:

\begin{equation}
    \tau_{\lambda}^{\mathrm{cont}}(z)
    = \int_z^{\infty}
      \chi_{\lambda}^{\mathrm{cont}}(z')\,dz'.
\end{equation}

U diskretnoj formi:

\begin{equation}
    \tau_{\lambda,i}^{\mathrm{cont}}
    = \tau_{\lambda,i+1}^{\mathrm{cont}}
      + \chi_{\lambda,i}^{\mathrm{cont}}\,(z_{i+1}-z_i).
\end{equation}

Kao i za linije pretpostavićemo LTR za funkciju izvora, pa nam ne treba eksplicitno koeficijent emisije.

\begin{equation}
    S_{\lambda}^{\mathrm{cont}} = B_{\lambda}(T).
\end{equation}

\subsubsection*{Rešavanje JPZ}

Kao i za liniju, koristimo formalno rešenje:

\begin{equation}
    I_{\lambda,i}
    = I_{\lambda,i+1}\, e^{-(\tau_{i+1}-\tau_i)}
      + S_{\lambda,i}\left(1 - e^{-(\tau_{i+1}-\tau_i)}\right).
\end{equation}

Na dnu atmosfere:

\begin{equation}
    I_{\lambda}(\tau_{\max})
    = B_{\lambda}(T(\tau_{\max})).
\end{equation}

Na kraju izračunajte izlazi intenzitet u kontinuumu na nekom razumnom intervalu talasnih dužina (recimo od 100 do 1000 nm) i proverite da li ima fizički smisao. Npr, da li vidite ikakav Balmerov i Pašenov skok.

\subsection{Korak 4: Populacije nivoa}

Izaberite spektralnu liniju koju ćete sintetisati.  Ne mora biti linija vodonika ili magnezijuma, mada je lakše uzeti neku od njih.  Na primer, Mg\,I b2 linija na 517.2\,nm.

Potrebni podaci (energije nivoa, statističke težine $g=2J+1$) mogu se preuzeti sa:

\url{https://physics.nist.gov/PhysRefData/ASD/lines_form.html}

Bolcmanova formula za populacije nivoa je:

\begin{equation}
    n_{j,i} = n_j\, g_i\,
             \frac{\exp(-E_i/kT)}{Z_j(T)}.
\end{equation}

Prikažite kako populacije vaša dva omiljena nivoa zavise od visine. 

\subsection{Korak 5: Neprozračnost u liniji}

U LTE važi $S_\nu = B_\nu(T)$. Fokusiramo se na:

\begin{equation}
    \chi_\nu = (n_l B_{lu} - n_u B_{ul})
               \frac{h\nu}{4\pi}\,\phi(\nu).
\end{equation}

Sa NIST baze preuzmite $A_{ul}$ i iz njega izvedite $B_{ul}$ i $B_{lu}$.


\subsection{Korak 6: Apsorpcioni profil linije}

Sada određujemo zavisnost neprozračnosti od frekvencije/talasne dužine.  
Izaberite opseg talasnih dužina i za svaku dubinu izračunajte:

\begin{itemize}
    \item Doplerovu širinu iz mase atoma i temperature
    \item Damping frekvenciju (možete uzeti $\Gamma=A_{ul}$)
    \item Zatim, tzv. bezdimenzionu frekvenciju: $u = (\nu-\nu_0)/\Delta\nu_D$
    \item Damping parametar $a=\Gamma/(4\pi\Delta\nu_D)$
    \item Voigtov profil $\phi(u, a)$
\end{itemize}

Zatim izračunajte:

\begin{equation}
    \chi_\nu = (n_l B_{lu} - n_u B_{ul})
               \frac{h\nu}{4\pi}\phi(\nu).
\end{equation}

\subsection{Korak 7: Optička dubina u liniji i funkcija izvora}

\textbf{Ukupna neprozračnost je zbir neprozračnosti u kontinuumu i liniji}

Integracijom neprozračnosti dobijamo ukupno optičku dubinu. Funkcija izvora u LTE je $B_\nu(T)$.

Prikažite $S_\lambda(\tau_\lambda)$ za više talasnih dužina radi provere.

\subsection{Korak 8: Formalno rešenje jednačine prenosa}

Iskoristite isto numeričko rešenje kao i gore, a i na času da izračunate spektar vaše linije! Prikažite izlazni spektar kao funkciju talasne dužine.

\subsection{Diskusija}

Kada sve proradi, možete eksperimentisati sa:

\begin{itemize}
    \item promenom energije donjeg nivoa,
    \item uključivanjem realističnijeg sudarnog širenja,
    \item uvođenjem polja brzina 
    \item izborom različitih atoma 
    \item promenom atmosferskog modela (možete podići ili spustiti temperaturu)
\end{itemize}

\section{Predaja}

\begin{itemize}
    \item Kratak izveštaj (cca 5-7 strana) sa metodologijom i plotovima.
    \item Rok: 31.\ decembar 2025.\ u 23:59.
    \item Python kod (opciono, samo ako nešto ne radi).
\end{itemize}

\end{document}
