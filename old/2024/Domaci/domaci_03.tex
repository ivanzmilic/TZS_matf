\documentclass[12pt]{article}
\usepackage{amsmath}
\usepackage{amssymb}
\usepackage{amsfonts}
\usepackage[colorlinks=true]{hyperref}
\usepackage{graphicx}
\usepackage[T1]{fontenc}
\usepackage[utf8]{inputenc}
\usepackage[english,main=serbian,provide=*]{babel}
%\selectlanguage{serbian}

\title{Prvi Doma\'{c}i zadatak}
\author{TZS}
\date{\today}

\begin{document}
\maketitle

U izradi doma\'{c}eg zadatka se mo\v{z}ete konsultovati medjusobno i sa mnom. Svaki doma\'{c}i koji predajete, medjutim, mora biti samostalno napisan. 

\textbf{Rok za predaju ovog doma\'{c}eg zadatka je petak 31.12.2024.}

\section{Setup i atmosferski model}

Cilj ovog zadatka je da sintetizujete profil spektralne linije u lokalnoj termodinamičkoj ravnoteži (LTE) iz datog atmosferskog modela (u ovom slučaju poznatog FALC modela iz rada Averett et al.\ 1993). Izračunaćete populacije atomskih nivoa koristeći Bolcmanovu i Sahinu jednačinu, zatim neprozračnost i funkciju izvora, i konačno rešiti JPZ. U procesu ćete takođe implementirati realističnu neprozračnost u kontinuumu.



Koristite dati jednodimenzioni atmosferski model \texttt{falc.dat}. Učitajte ga pomoću \texttt{numpy.loadtxt} ili slične funkcije. Model ima više kolona, od kojih nam trebaju:

\begin{itemize}
    \item[0:] Log optičke dubine u kontinumu na 500\,nm: $\log \tau_{500}$
    \item[1:] Visina $z$ u cm
    \item[2:] Temperatura $T(z)$ u K
    \item[3:] Pritisak gasa $P_{\mathrm{gas}}(z)$ u dyn/cm$^2$
    \item[4:] Elektronski pritisak $n_e(z)$ u dyn/cm$^2$
\end{itemize}

\textbf{Obratite pažnju na jedinice!}

\section{Zadatak}

\subsection{Korak 1: Jonizacija vodonika}

Kao uvod, iskoristite Sahinu jednačinu \textbf{samo za vodonik} (kao na času) da izračunate jonizaciono stanje vodonika kao funkciju visine. Iz toga ćete dobiti i elektronsku koncentraciju. Uporedite dobijene rezultate sa vrednostima iz modela (plotujte u log skali!).

Sahina jednačina glasi (uvek proverite ovo):

\begin{equation}
    \frac{n_{j+1} n_e}{n_j}
    = \frac{2 Z_{j+1}}{Z_j}
      \left( \frac{2\pi m_e kT}{h^2} \right)^{3/2}
      \exp\left( -\frac{E_j}{kT} \right).
\end{equation}

Ovde je $Z$ particiona funkcija. U dubokim slojevima dobićete dobro slaganje sa modelom, ali u višim slojevima naši rezultati pokazuju pad elektrona ispod vrednosti datih u FALC modelu.

\textbf{Zašto se ovo dešava?}

\emph{Napomena: još uvek ne objašnjavamo zašto su naši rezultati iznad modela u najvišim slojevima.}

\subsection{Korak 2: Dodavanje magnezijuma kao dodatnog donora elektrona}

Da bismo popravili rezultate u srednjoj atmosferi, uvodimo dodatni element: magnezijum (pomnožićem njegovu zastupljenost sa 2 da obračunamo i doprinos gvožđa, umesto da rešavamo problem za još jedan element).

Ponovite Saha proračun za sistem H i Mg.  Ovo sada zahteva rešavanje \textbf{nelinearnog sistema jednačina} pomoću, na primer, \texttt{scipy.optimize.fsolve}.

Koristite sledeće parametre:

\begin{itemize}
    \item Energija jonizacije Mg\,I $\rightarrow$ Mg\,II: 7.646 eV
    \item Energija jonizacije Mg\,II $\rightarrow$ Mg\,III: 15.035 eV
    \item Zastupljenost Mg: $3.8\times 10^{-5}$  
          \textbf{Pomnožiti sa 2 da se uračuna i Fe}
    \item Particioni funkcionali: $Z_{\mathrm{Mg\,I}}=2$, $Z_{\mathrm{Mg\,II}}=1$, $Z_{\mathrm{Mg\,III}}=1$
\end{itemize}

Ponovo izračunajte elektronsku gustinu. Trebalo bi da dobijete znatno bolje slaganje sa modelom. Možete dodatno da varirate zastupljenost Mg ili particionu funkciju za Mg+ da dobijete još bolje rešenje.

\textbf{Napomena:} Često je dovoljno uzeti samo Mg\,I–Mg\,II i zanemariti Mg\,III.

\subsection{Korak 3: Neprozračnost i intenzitet u kontinuumu}

Pre sinteze profila linije potrebno je odrediti \textbf{neprozračnost u kontinuumu} i onda rešiti JPZ za kontinuum.

Dominantni izvori neprozračnosti u vidljivom domenu su negativni jon vodonika $\mathrm{H}^-$ i neutralni vodonik.  Kontinualni opacitet definišemo kao:

\begin{equation}
    \chi_\lambda^{\mathrm{cont}}
    = \chi_\lambda^{\mathrm{H^-}} + \chi_\lambda^{\mathrm{H}}.
\end{equation}


Da biste obračunali ova dva doprinosa, potrebno je da iskoristite jednačine koje se pojavljuju knjizi Gray-a i opisuju neprozračnost negativnog jona vodonika i neutralnog vodonika. Ako nemate pristup knjizi, javite se meni.

\subsubsection*{Optička dubina u kontinuumu}

Kao i uvek:

\begin{equation}
    \tau_{\lambda}^{\mathrm{cont}}(z)
    = \int_z^{\infty}
      \chi_{\lambda}^{\mathrm{cont}}(z')\,dz'.
\end{equation}

U diskretnoj formi:

\begin{equation}
    \tau_{\lambda,i}^{\mathrm{cont}}
    = \tau_{\lambda,i+1}^{\mathrm{cont}}
      + \chi_{\lambda,i}^{\mathrm{cont}}\,(z_{i+1}-z_i).
\end{equation}

\subsubsection*{Kontinualni izvorni funkcional}

\begin{equation}
    S_{\lambda}^{\mathrm{cont}} = B_{\lambda}(T).
\end{equation}

\subsubsection*{Kontinualni intenzitet}

Kao i za liniju, koristimo formalno rešenje:

\begin{equation}
    I_{\lambda,i}
    = I_{\lambda,i+1}\, e^{-(\tau_{i+1}-\tau_i)}
      + S_{\lambda,i}\left(1 - e^{-(\tau_{i+1}-\tau_i)}\right).
\end{equation}

Na dnu atmosfere:

\begin{equation}
    I_{\lambda}(\tau_{\max})
    = B_{\lambda}(T(\tau_{\max})).
\end{equation}

Na kraju prikažite kontinualni intenzitet i proverite da li ima fizički smisao.

\subsection{Korak 3: Populacije nivoa}

Izaberite spektralnu liniju koju ćete sintetisati.  Ne mora biti linija vodonika ili magnezijuma, mada je lakše uzeti neku od njih.  Na primer, Mg\,I b2 linija na 517.2\,nm.

Potrebni podaci (energije nivoa, statističke težine $g=2J+1$) mogu se preuzeti sa:

\url{https://physics.nist.gov/PhysRefData/ASD/lines_form.html}

Bolcmanova formula za populacije nivoa glasi:

\begin{equation}
    n_{j,i} = n_j\, g_i\,
             \frac{\exp(-E_i/kT)}{Z_j(T)}.
\end{equation}

Prikažite promenu nivoa populacija sa visinom.

\subsection{Korak 4: Opacitet linije i izvorni funkcional}

U LTE važi $S_\nu = B_\nu(T)$. Fokusiramo se na opacitet:

\begin{equation}
    \chi_\nu = (n_l B_{lu} - n_u B_{ul})
               \frac{h\nu}{4\pi}\,\phi(\nu).
\end{equation}

Sa NIST baze preuzmite $A_{ul}$ i iz njega izvedite $B_{ul}$ i $B_{lu}$ uz relacije sa časa.




\subsection{Korak 6: Profil linije}

Sada određujemo zavisnost od frekvencije/talasne dužine.  
Izaberite opseg talasnih dužina i za svaku dubinu računajte:

\begin{itemize}
    \item Doplerovu širinu iz mase i temperature
    \item Damping frekvenciju (možete uzeti $\Gamma=A_{ul}$)
    \item Bezdimenzioni pomak $u = (\nu-\nu_0)/\Delta\nu_D$
    \item Damping parametar $a=\Gamma/(4\pi\Delta\nu_D)$
    \item Voigtov profil $\phi(\nu)$
\end{itemize}

Zatim izračunajte opacitet linije:

\begin{equation}
    \chi_\nu = (n_l B_{lu} - n_u B_{ul})
               \frac{h\nu}{4\pi}\phi(\nu).
\end{equation}

\subsection{Korak 7: Optička dubina linije i izvorni funkcional}

Integracijom opaciteta dobijamo optičku dubinu linije.  
Izvorni funkcional u LTE je $B_\nu(T)$.

\textbf{Ukupna optička dubina je zbir kontinualne i linijske optičke dubine.}

Prikažite $S_\lambda(\tau_\lambda)$ za više talasnih dužina radi provere.

\subsection{Korak 8: Formalno rešenje jednačine prenosa}

Radiativni prenos glasi:

\begin{equation}
    \frac{dI_\nu}{d\tau_\nu} = I_\nu - S_\nu.
\end{equation}

Koristimo formalno rešenje:

\begin{equation}
    I(\tau_i)
    = I(\tau_{i+1}) e^{-(\tau_{i+1}-\tau_i)}
    + S(\tau_i)\left( 1 - e^{-(\tau_{i+1}-\tau_i)} \right).
\end{equation}

Na dnu atmosfere:

\begin{equation}
    I_\nu(\tau_{\max}) = B_\nu(T(\tau_{\max})).
\end{equation}

Prikažite emergentni spektar kao funkciju talasne dužine.

\subsection{Diskusija}

Kada sve proradi, možete eksperimentisati:

\begin{itemize}
    \item promenom donjeg nivoa,
    \item uključivanjem sudarnog proširenja,
    \item uvođenjem brzina radi linijske asimetrije,
    \item izborom različitih atomskih vrsta,
    \item promenom atmosferskog modela.
\end{itemize}

\section{Predaja}

\begin{itemize}
    \item Kratak izveštaj (cca 5 strana) sa metodologijom i grafikonom.
    \item Rok: 31.\ decembar 2025.\ u 23:59.
    \item Python kod (opciono, samo ako nešto ne radi).
\end{itemize}

\end{document}
