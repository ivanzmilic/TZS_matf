\documentclass[12pt]{article}
\usepackage{amsmath}
\usepackage{amssymb}
\usepackage{amsfonts}
\usepackage[colorlinks=true]{hyperref}


\title{TZS Ve\v{z}be: \v{C}as 2, 21/10/2022}
\author{Ivan Mili\'{c}}
\date{\today}

\begin{document}
\maketitle

\section{Zadatak 1}

Zemljina atmosfera apsorbuje oko 25\% zra\v{c}enja izvora koji je u zenitu.
\begin{itemize}
    \item Koja je ukupna opti\v{c}ka dubina (debljina) Zemljine atmosfere?
    \item Pod pretpostavkom da je temperatura atmosfere konstantna, i da iznosi oko 300\,K, odredite skalu visine za zemljinu atmosferu.
    \item Procenite koncentraciju \v{c}estica vazduha na povr\v{s}ini. 
    \item Iz rezultata prethodnih delova, izvedite prose\v{c}an efikasni presek za apsorpciju na \v{c}esticama vazduha. (Iako je vazduh me\v{s}avina gasova, pretpostavi\'{c}emo da se svi pona\v{s}aju isto.)
    \item Kako se menja opti\v{c}ka dubina koju ``vidi'' svetlost, ako izvor nije u zenitu? Dati opisan odgovor.
\end{itemize}

\section{Zadatak 2}

Na pro\v{s}lom \v{c}asu smo izveli da je tzv. formalno re\v{s}enje jedna\v{c}ine prenosa:

\begin{equation}
I_\lambda^+ = I_\lambda^0 e^{-\tau_\lambda}+ \int_0^{\tau_\lambda} S(t) e^{-t} dt
\end{equation}

\begin{itemize}
    \item Za fiksnu talasnu du\v{z}inu i konstantnu funkciju izvora, re\v{s}ite jedna\v{c}inu prenosa i uporedi ulazni i izlazi intenzitet.
\item Kako bi izgledalo formalno re\v{s}enje za polubeskona\v{c}nu zvezdanu atmosferu? (Donja granica ima efektivno beskona\v{c}nu opti\v{c}ku dubinu)?
\item Milne-Eddingtonova aproksimacija pretpostavlja da funkcija izvora u polubeskona\v{c}noj atmosferi raste sa opti\v{c}kom dubinom. Nadjite izlazni intenzitet za takvu atmosferu. Prodiskutujte re\v{s}enje.

\end{itemize}

\section{Zadatak 3}
Neprozra\v{c}nost, pa samim tim i opti\v{c}ka dubina, zavise od talasne du\v{z}ine. Pretpostavite neku zavisnost $\tau_\lambda$ i analizirajte $I_\lambda$ za primere iz zadatka 2.

\section{Zadatak 4}
Jednostavan model formiranja spektralnih linija pretpostavlja sloj gasa fiksne temperature kroz koji prolazi zracenje koje je emitovala zvezda ispod, za koju pretpostavljamo da zra\v{c}ki kao crno telo. Ako uzmemo da, u spektralnoj liniji, opti\v{c}ka dubina zavisi od talasne du\v{z}ine kao neka Gausovska funkcija, isplotujte (koriste\'{c}i npr python) izlazni spektar zra\v{c}enja za razli\v{c}ite odnose temperatura zvezde i atmosfere, kao i za razli\v{c}ite opti\v{c}ke dubine u centru linije (detaljno \'{c}emo ove teme pokriti za nekoliko nedelja, ali dobro je da ve\'{c} razmi\v{s}ljamo o linijama!)
\end{document}
