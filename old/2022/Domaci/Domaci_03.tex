\documentclass[12pt]{article}
\usepackage{amsmath}
\usepackage{amssymb}
\usepackage{amsfonts}
\usepackage[colorlinks=true]{hyperref}
\usepackage{graphicx}

\title{Tre\'{c}i Doma\'{c}i zadatak}
\author{TZS}
\date{\today}

\begin{document}
\maketitle

U izradi doma\'{c}eg zadatka se mo\v{z}ete konsultovati medjusobno i sa mnom. Svaki doma\'{c}i koji predajete, medjutim, mora biti samostalno napisan. 

\textbf{Rok za predaju ovog doma\'{c}eg zadatka je petak 06.01.2023. Mo\v{z}ete predati doma\'{c}i u .pdf-u!}

\section*{Zadatak 1}

(Svaki deo nosi po 5 poena)

Razmatramo interval talasnih du\v{z}ina gde imamo apsorpciju i emisiju usled spektralne linije i usled kontinuuma. Pretpostavi\'{c}emo da je neprozra\v{c}nost u kontinuumu konstantna sa talasnom du\v{z}inom i jednaka $\chi^c$. Pretpostavimo da je kontinuum u LTR. Sa druge strane, defini\v{s}imo \emph{funkciju izvora u liniji} kao odnos koeficijenata emisije i apsorpcije u liniji. Dakle:
\begin{equation}
S^L = \frac{j_\lambda^L}{\chi_\lambda^L},
\end{equation}
gde ova funkcija izvora, generalno, ne mora biti u LTR. Kada re\v{s}avamo jedna\v{c}inu prenosa, medjutim, koristimo \emph{ukupnu} fukciju izvora koja je jednaka odnosu ukupnih koeficijenata emisije i apsorpcije (kontinuum + linija).

\begin{itemize}
\item Pokazati da je ukupna funkcija izvora:
\begin{equation}
S_\lambda = \frac{\chi^L_\lambda}{\chi} S^L + \frac{\chi^c}{\chi} B.
\end{equation}

Obratiti pa\v{z}nju da smo napisali $S_\lambda$. Za\v{s}to, ako ni funkcija izvora u liniji ni u kontinuumu ne zavise od talasne du\v{z}ine?

\item Pod pretpostavkom da je $S_L = B = a+b\tau$ (Milne-Eddingtonova atmosfera). I da je neprozra\v{c}nost u liniji:
\begin{equation}
\chi^L_\lambda = \chi^L \phi_\lambda,
\end{equation}
te da je: 
\begin{equation}
\frac{\chi^L}{\chi^c} = \eta,
\end{equation}
pokazati da je izraz za izlazni intenzitet:
\begin{equation}
I_\lambda = a + \mu\frac{b}{1+\eta \phi_\lambda}.
\end{equation}

\item Pretpostaviti da apsorpcioni profil linije ima gausovski oblik, sa zadatom doplerovom \v{s}irinom ($\Delta \lambda_D$) i centrom. Prikazati, koriste\'{c}i program za plotovanje po \v{z}elji, uticaj parametara $\eta$ i $\Delta \lambda _D$ na oblik linije.
\end{itemize}

\section*{Zadatak 2}

(5 poenta)

Neke linije u spektru Sunca imaju emisiju u centru (npr. Mg II h \& k linije), dok neke nemaju (ve\'{c}ina ostalih). Objasniti za\v{s}to.
\end{document}
