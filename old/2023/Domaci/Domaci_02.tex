\documentclass[12pt]{article}
\usepackage{amsmath}
\usepackage{amssymb}
\usepackage{amsfonts}
\usepackage[colorlinks=true]{hyperref}
\usepackage{graphicx}

\title{Drugi Doma\'{c}i zadatak}
\author{TZS}
\date{\today}

\begin{document}
\maketitle

U izradi doma\'{c}eg zadatka se mo\v{z}ete konsultovati medjusobno i sa mnom. Svaki doma\'{c}i koji predajete, medjutim, mora biti samostalno napisan. 

\textbf{Rok za predaju ovog doma\'{c}eg zadatka je petak 12.01.2024.}

\section*{Zadatak 1}

Re\v{s}iti Milneov problem tzv. metodom diskretnih ordinata. Metoda diskretnih ordinata pretpostavlja da diskretizujemo intenzitet u nekoliko pravaca, pa se onda integral po uglovima svodi na sumu. Pretpostaviti da razmatramo jedan ulazni i jedan izlazni pravac, u pravcima $\mu=-1/\sqrt{3}$ i $\mu=1/\sqrt{3}$. 

\section*{Zadatak 2}

Razmatrajmo formiranje spektralne linije u Milne-Eddingtonovoj atmosferi ($S=a+b \tau_c$), gde je linija okarakterisana ja\v{c}inom linije, $\eta$, koja je jednaka odnosu izmedju neprozra\v{c}nosti linije u centru i neprozra\v{c}nosti u kontinuumu:
\begin{equation}
\chi_\lambda^L = \eta \phi_\lambda \chi_c.
\end{equation}
Dakle:
\begin{equation}
\tau_\lambda = (1+\eta \phi_\lambda) \tau_c.
\end{equation}
Ovde \'{c}emo pretpostaviti da je profil linije dat kao:
\begin{equation}
\phi_\lambda = \frac{1}{\sqrt{\pi}} e^{-(\lambda-\lambda_0)^2/\Delta\lambda_D^2}.
\end{equation}
Poka\v{z}ite da se izlazni intenzitet u liniji mo\v{z}e napisati kao: 
\begin{equation}
I_\lambda = a + \frac{b}{1+\eta \phi_\lambda}.
\end{equation}
Ekvivalentna \v{s}irina linije je definisana kao:
\begin{equation}
EW = \int 1 - \frac{I_\lambda}{I_c} d\lambda.
\end{equation}

Za veliki opseg $\eta$ i dato $a,b,\lambda_0, \Delta\lambda_D$ izra\v{c}unati EW i nacrtati zavisnost $EW(\eta)$.

\section*{Zadatak 3}

Neke linije u spektru Sunca imaju emisiju u centru (npr. Mg II h \& k linije), dok neke nemaju (ve\'{c}ina ostalih). Objasniti za\v{s}to.


\end{document}
