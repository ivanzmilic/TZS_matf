\documentclass[12pt]{article}
\usepackage{amsmath}
\usepackage{amssymb}
\usepackage{amsfonts}
\usepackage[colorlinks=true]{hyperref}


\title{TZS Ve\v{z}be: 6. Nedelja, 01/12/2022}
\author{Jasmina Horvat, Ivan Mili\'{c}}
\date{\today}

\begin{document}
\maketitle

\section{Zadatak 1}
Pokazati da za atmoferu gde neprozra\v{c}nost ne zavisi od talasne du\v{z}ine va\v{z}i ravnote\v{z}a zra\v{c}enja u obliku:
\begin{equation}
J = S 
\end{equation}
gde su $J$ i $S$ srednji intenzitet i funkcija izvora, integraljeni po talasnim du\v{z}inama, respektivno. 

\section{Zadatak 2}
Pokazati da je za atmosferu gde neprozra\v{c}nost zavisi od talasne du\v{z}ine, ravnote\v{z}a zra\v{c}enja implicira da je ukupan fluks (na svim talasnim du\v{z}inama) konstantan sa visinom. 

\section{Zadatak 3}
Kori\v{s}\'{c}enjem prve Eddingtonove aproskimacije ($J = 3K$), pokazati da ravnote\v{z}a zra\v{c}enja prirodno vodi ka pretpostavci Milne-Eddingtonove atmosfere ($S = a+b \tau$).

\section{Zadatak 4}
Kori\v{s}\'{c}enjem druge Eddingtonove aproskimacije (izlazno zra\v{c}enje je izotropno), izvesti Eddingtonovo re\v{s}enje Milneovog problema. 

\section{Zadatak 5}
Pod pretpostavkom da je struktura atmosfere u skladu sa Eddingtonovim re\v{s}enjem Milneovog problema, proceniti ja\v{c}inu Balmerovog skoka, ako je efikasna temperatura zvezde 8000 K a $\chi^-_{Balmer} = 10 \chi^+_{Balmer}$, a $\overline{\chi} = \chi^+$. Dovoljno je opisno uraditi zadatak (tj. bez zamene brojki).

\section{Zadatak 6}
Ako se gas sastoji samo od protona, elektrona i neutralnog vodonika, i ako su date \emph{elektronska} koncentracija gasa (ili elektronski pritisak) i temperatura, izra\v{c}unati koncentraciju protona i neutralnog vodonika, kao i ukupan pritisak gasa. Prodiskutovati razlike u odnosu na slu\v{c}aj kada je bio dat pritisak gasa. 

\section{Zadatak 7}
Postaviti prethodni zadatak ako se gas sastoji od protona, elektrona, neutralnog vodonika, neutralnog helijuma i jednom i dvaput jonizovanog helijuma. 

\section{Zadatak 8}
Za plazmu temperature 4000\,K i fotosferskog pritiska, uporediti koncentraciju vodonika na nivou $n=2$ i koncentraciju gvo\v{z}dja (zastupljenost gvo\v{z}dja je oko $3\times 10^{-5}$ u odnosu na vodonik).





\end{document}

