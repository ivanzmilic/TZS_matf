\documentclass[12pt]{article}
\usepackage{amsmath}
\usepackage{amssymb}
\usepackage{amsfonts}
\usepackage[colorlinks=true]{hyperref}


\title{TZS Ve\v{z}be: 6. Nedelja, 17/11/2023}
\author{Jasmina Horvat, Ivan Mili\'{c}}
\date{\today}

\begin{document}
\maketitle

\section{Zadatak 1}
\begin{itemize}
    \item Pod pretpostavkom da je gas sa\v{c}injen samo od vodonika (neutralnog i pozitivnog jona vodonika, kao i elektrona), pokazati kako se, na osnovu $T$ i $p$, mogu izra\v{c}unati koncentracije neutralnog vodonika, protona i elektrona (ovo je ve\'{c} uradjeno).
    \item Sada, pretpostaviti da je koncentracija negativnog jona vodonika mnogo manja od koncentracije neutralnog vodonika, protona i neutrona, pa da ako razmatramo i $H-$, koncentracija ostalih \v{c}estica se ne menja. 
    \item Za temperaturu 6500\,K i pritisak $10^4$\,Pa proceniti koncentraciju negativnog jona vodonika. 
    \item Uporediti istu sa koncetracijom neutralnog vodonika ekscitovanog na $n=2$ i $n=3$ nivoe. 
\end{itemize}

\section{Zadatak 2}
Razmatramo Milne-Eddingtonovu, sivu atmosferu atmosferu: $S=a+b\tau$. Ispitajte da li za ovakvu atmosferu va\v{z}i ravnote\v{z}a zra\v{c}enja $\tau >> 1$. 

Ako je dat izlazni fluks i ako ravnote\v{z}a zra\v{c}enja va\v{z}i i na povr\v{s}ini, odrediti $a$ i $b$.

\section{Zadatak 3}
Sada \'{c}emo uvesti tzv. Schwarzschild-Milne operatore. To su operatori koji preslikavaju funkciju izvora $S(\tau)$ u srednji intenzitet, fluks i K integral. Oni u sebi, implicitno, sadr\v{z}e re\v{s}enje jedna\v{c}ine prenosa i odgovaraj\'{c}u integraciju po uglovima. 

Izvedite izraze za ova tri operatora i uo\v{c}ite pojavu tzv. eksponencijalnih integrala:
\begin{equation}
E_n(x) = \int_1^{\infty} \frac{e^{-xt}}{t^n} dt.
\end{equation}

Trebalo bi da dobijete da operatori imaju oblik: 

\begin{equation}
\int_0^{\infty} E_n(|\tau-t|) S(t) dt.
\end{equation}

Eksponencijalni operator se ovde pona\v{s}a kao \emph{kernel}. Koriste\'{c}i Python ili sli\v{c}no isplotujte ove funkcije i prodiskutujte njihovo pona\v{s}anje. 

\section{Zadatak 4}

Jednostavan model formiranja spektralnih linija pretpostavlja sloj gasa fiksne temperature kroz koji prolazi zracenje koje je emitovala zvezda ispod, za koju pretpostavljamo da zra\v{c}i kao crno telo. Ako uzmemo da, u spektralnoj liniji, opti\v{c}ka dubina zavisi od talasne du\v{z}ine kao neka Gausova funkcija, isplotujte (koristeći npr python) izlazni spektar zračenja za različite odnose temperatura zvezde i atmosfere, kao i za različite optičke dubine u centru linije (detaljno ćemo ove teme pokriti za nekoliko nedelja, ali dobro je da već razmišljamo o linijama!). Obratite pa\v{z}nju da linija lokalno apsorbuje po Gausovoj raspodeli ali izlazni oblik linije nije Gausova funkcija


\end{document}

