\documentclass[12pt]{article}
\usepackage{amsmath}
\usepackage{amssymb}
\usepackage{amsfonts}
\usepackage[colorlinks=true]{hyperref}


\title{TZS Ve\v{z}be: 4. Nedelja, 03/11/2023}
\author{Jasmina Horvat, Ivan Mili\'{c}}
\date{\today}

\begin{document}
\maketitle

\section{Zadatak 1}
Posmatramo Sunce teleskopom pre\v{c}nika 1 m. Fokusiramo se na ``piksel'' oblika kvadrata u centru Sun\v{c}evog diska, stranice 100\,km. Pod pretpostavkom da povr\v{s}ina Sunca zra\v{c}i kao apsolutno crno telo (ina\v{c}e ne zra\v{c}i), izra\v{c}unajte broj fotona koji padne na teleskop u jednoj sekundi na talasnoj du\v{z}ini 500\,nm, u intervalu talasnih du\v{z}ina \v{s}irokom 2\,pm.

\section{Zadatak 2}
Milne-Eddingtonova aproksimacija pretpostavlja da funkcija izvora u polubeskona\v{c}noj atmosferi raste linearno sa nekom, referentnom opti\v{s}kom dubinom $\tau$:
\begin{equation}
S = a + b\tau
\end{equation}
\begin{itemize}
\item Neprozra\v{c}nost, pa samim tim i opti\v{c}ka dubina, zavise od talasne du\v{z}ine. Pretpostavimo na je odnos neprozra\v{c}nosti na nekoj talasnoj du\v{z}ini $\lambda$ i referentne neprozra\v{c}nosti konstantan sa dubinom i jednak $r_\lambda$.
\begin{equation}
\frac{\chi_\lambda}{\chi} = r_\lambda. 
\end{equation}
Pokazati da je:
\begin{equation}
\frac{\tau_\lambda}{\tau} = r_\lambda. 
\end{equation}.

\item Pokazati da je izlazni intenzitet na talasnoj du\v{z}ini $\lambda$ jednak:
\begin{equation}
I_\lambda = a + \frac{b}{r_\lambda}
\end{equation}

\item Ubedite se da je intenzitet manji za neprozra\v{c}nije talasne du\v{z}ine, ali da ne dosti\v{z}e nulu.
\end{itemize}

\section{Zadatak 3}
\begin{itemize}
    \item Pod pretpostavkom da je gas sa\v{c}injen samo od vodonika (neutralnog i pozitivnog jona vodonika, kao i elektrona), pokazati kako se, na osnovu $T$ i $p$, mogu izra\v{c}unati koncentracije neutralnog vodonika, protona i elektrona. 
    \item Izra\v{c}unati stepen jonizacije vodonika ($n(H+) / (n(H_0) + n(H+))$ za T = 8000\,K i za dva razli\v{c}ita pritiska, $10^4$Pa i $1$\,Pa.
    \item Trebalo bi da ste dobili da je stepen jonizacije ve\'{c}i za drugi slu\v{c}aj, kako to mo\v{z}emo da objasnimo intuitivno?
    \item Za prvi slu\v{c}aj ($p=10^4$\,Pa), uporediti koncentraciju elektrona sa koncentracijom vodonika ekscitovanih na $n=2$.
\end{itemize}


\end{document}

