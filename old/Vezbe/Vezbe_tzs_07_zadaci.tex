\documentclass[12pt]{article}
\usepackage{amsmath}
\usepackage{amssymb}
\usepackage{amsfonts}
\usepackage[colorlinks=true]{hyperref}


\title{TZS Ve\v{z}be: \v{C}as 7, 25/11/2022 - Razni zadaci}
\author{Ivan Mili\'{c}}
\date{\today}

\begin{document}
\maketitle

\section*{Zadatak 1}

Pokazati da se upotrebom srednjeg koeficijenta neprozr\v{c}nosti po fluksu dobija jednostavan izraz za gradijent pritiska zra\v{c}enja:
\begin{equation}
\frac{dp_{\rm zr}}{d \tau} = \frac{\sigma}{c} T_{\rm eff}^4 
\end{equation}

\section*{Zadatak 2}
Poku\v{s}ajte da procenite uticaj neutralnih metala (npr. gvo\v{z}dja) na apsorpciju u kontinuumu za zvezde Sun\v{c}evog tipa. 

\section*{Zadatak 3}
Re\v{s}iti Milneov problem metodom diskretnih ordinata (tj. diskretizacijom po uglovima). Pretpostaviti da su pravci diskretizovani tako da imamo jedan ulazni i jedan izlazni pravac. Koristiti Gausovu kvadraturnu formulu, koja za ovaj slu\v{c}aj ima \v{c}vorove u $\pm \frac{1}{\sqrt{3}}$. 

\emph{Prodiskutovati Gausovu kvadraturu po\v{s}to \'{c}e biti va\v{z}na za nastavak kursa}.

\section*{Zadatak 4}
Izra\v{c}unati $J= \Lambda[S]$ za pretpostavku linearne funkcije izvora (M-E atmosfera). Da li Milne-Eddingtonova atmosfera mo\v{z}e da bude ta\v{c}no re\v{s}enje Milneovog problema?

\end{document}
